\documentclass{article}
\usepackage[utf8]{inputenc}   % 支持 UTF-8 編碼
\usepackage{amsmath, amssymb} % 支持數學符號
\usepackage{graphicx}         % 插入圖片
\usepackage{hyperref}         % 超鏈接
\usepackage{caption}          % 圖表標題
\usepackage{subcaption}       % 子圖表
\usepackage{listings}         % 代碼高亮
\usepackage{xcolor}           % 顏色
\usepackage{geometry}         % 頁面設置
\usepackage{fancyhdr}         % 頁眉頁腳
\usepackage{booktabs}         % 美化表格
\usepackage{multicol}         % 多欄排版
\usepackage{tikz}             % 繪圖
\usepackage{enumitem}         % 自定義列表
\usepackage{float}

%中文輸入配置模塊
%如果報錯不能編譯,請至Options -> Configure TeXstudio後,選Build,將Default Compiler設置爲支持xeCJK包的XeLatex或LuaLaTeX



\ifXeTeX	%若使用XeTex編譯器
\usepackage{xeCJK}	% 支持中文輸入
\usepackage{fontspec}	% 允許更換字體
\setmainfont[Path=./fonts/]{TW-Kai-98-1.ttf} %楷體
%\setmainfont[Path=./fonts/]{TW-Sung-98-1.ttf} %宋體
%\newfontfamily\song{SimSun}
\fi

\ifLuaTeX
\usepackage{luatexja-fontspec}	% LuaLaTeX支持的中文包
\usepackage{fontspec}	% 允許更換字體
\setmainfont[Path=./fonts/]{TW-Sung-98-1.ttf} %宋體
\fi


% 頁面設置
\geometry{a4paper, margin=1in}

% 頁眉頁腳設置
\pagestyle{fancy}
\fancyhf{}
\fancyhead[L]{Your Name} %[位置][內容] 位置:E(偶數頁)O(奇數頁)LCR
\fancyhead[R]{\thepage}

\lstdefinelanguage{git}{
	morekeywords={git, init, clone, add, mv, reset, rm, bisect, grep, log, show, status, branch, checkout, commit, diff, merge, rebase, tag, fetch, pull, push, stash, remote, clean, reflog},
	sensitive=true,
	morecomment=[l]{//},
	%morestring=[b]", %b表示成對,意思是用成對的雙引號標示字串
	%moredelim=**[is][\color{purple}]{--}{ }, %以--開始以-結束的代碼標紫
}

% 代碼高亮設置
\lstset{
	language=git,
	basicstyle=\ttfamily\small,
	keywordstyle=\color{blue}\bfseries,
	commentstyle=\color{green},
	stringstyle=\color{red},
	showstringspaces=false,
	frame=single,
	numbers=left,
	numberstyle=\tiny\color{gray},
	breaklines=true,
	backgroundcolor=\color{lightgray!20},
	captionpos=b,
	%moredelim=[is][\color{green}]{<}{>}, %i表示界定符也會上色,s表示成對
}

% 用於縮短高亮指令
\newcommand{\rtext}[1]{\textcolor{red}{#1}}
\newcommand{\btext}[1]{\textcolor{blue}{#1}}

\newcommand{\yb}[1]{\colorbox{yellow}{#1}}

\newcommand{\code}[1]{\texttt{#1}}
\newcommand{\redcode}[1]{\textcolor{red}{\texttt{#1}}}

\newcommand{\gitcode}[1]{
	\begin{lstlisting}[language=c++]
		#1
	\end{lstlisting}
}

\title{Git使用筆記繁中版}
\author{盧永鈞}
\date{\today}

\begin{document}
	
\maketitle % 打印作者姓名,編輯日期

%\begin{abstract}
	
%\end{abstract}
	
\tableofcontents %打印目錄
	
\section{簡介}
這是簡介部分,簡要介紹文檔的背景和目的。Enslish font is ok

\section{基本操作}
\subsection{設置作者}
使用\texttt{config}指令設定作者和信箱
\begin{lstlisting}[language=git]
git config --global.user.name "<name>"
git config --global.user.email "<email>"
git config --list //檢視設定
\end{lstlisting}
也可以至家目錄的git設定檔\texttt{~/.gitconfig}修改
\\\\
將\texttt{global}替換成\texttt{local},可以爲每個專案可以設置獨立的作者
\subsection{設置指令縮寫}
\begin{lstlisting}[language=git]
git config --global alias.<指令別名> <指令>
git config --global alias.l "log --oneline --graph" //簡化命令
\end{lstlisting}
\subsection{初始化專案}
使用\texttt{init}指令使資料夾被Git控制
\begin{lstlisting}[language=git]
git init
\end{lstlisting}
\subsection{察看目錄狀態}
\begin{lstlisting}[language=git]
git status
\end{lstlisting}
\subsection{建立\texttt{commit}的標準流程}
先將想要追蹤的檔案加入暫存區 (Staging Area, index)
\begin{lstlisting}[language=git]
//將檔案加入暫存區 
git add <檔案>
//支持萬用字符,僅加入特定類型檔案
git add *.檔案後綴
//將所有檔案加入暫存區
git add --all
//在資料夾根目錄使用,等效於--all
git add .
\end{lstlisting}
然後就可以\texttt{commit},即把檔案加入儲存庫 (Repository)。
\begin{lstlisting}[language=git]
git commit -m "說明"
\end{lstlisting}
注意到,\texttt{commit}只會對加入暫存區的檔案作「備份」。然而,也可以在暫存區爲空的情況下\texttt{commit}
\begin{lstlisting}[language=git]
git commit --allow-empty -m "<說明>"
\end{lstlisting}
\subsection{察看\texttt{commit}歷史記錄}
使用\texttt{log}指令察看\texttt{commit}記錄
\begin{lstlisting}[language=git]
git log
git log --oneline
git log --oneline --graph
\end{lstlisting}
也可以用特定條件篩選\texttt{commit}記錄
\begin{lstlisting}[language=git]
//用作者篩選
git log --auhtor="作者"
//用commit內容篩選
git log --grep="關鍵字"
//用檔案內容篩選
git log -S "關鍵字"
//檢視特定檔案的commit記錄
git log <檔案>
//檢視特定檔案的修改記錄
git log -p <檔案>
//查看特定檔案每一行/特定行數的作者
git blame <檔案>
git blame -L <開始行數>,<結束行數> <檔案> 
\end{lstlisting}

\subsection{在Git中刪除檔案}
若使用linux自帶的的\texttt{rm},還得手動加入暫存區,用Git自帶的\texttt{rm}指令可以結合上述兩個操作:
\begin{lstlisting}[language=git]
git rm <要刪除的檔案>
\end{lstlisting}

\subsection{救回誤刪的檔案}
若只是檔案誤刪,使用\texttt{checkout}指令,從暫存區取用資料復原
\begin{lstlisting}[language=git]
git checkout <誤刪的檔案> 
git checkout . //救回所有刪除的檔案
\end{lstlisting}

\subsection{把檔案還原到特定commit}
將<檔案>還原到<數字>個版本前的\code{commit}
\begin{lstlisting}[language=git]
git checkout HEAD~<數字> <檔案>
\end{lstlisting}

\subsection{在Git中變更檔名}
\begin{lstlisting}[language=git]
	git mv <舊黨名> <新黨名>
\end{lstlisting}

\subsection{取消版控}
使用Git的\texttt{rm}指令,可以將檔案解除版控而不刪除
\begin{lstlisting}[language=git]
git rm <要刪的檔案> --cached
\end{lstlisting}

\section{修改\texttt{commit}與進階操作}
\subsection{修該最新\code{commit}訊息}
\begin{lstlisting}[language=git]
git commit --amend -m "新訊息"
\end{lstlisting}

\subsection{新增檔案到最新的\code{commit}}
\begin{lstlisting}[language=git]
//先加到緩存區
git add <要新增的檔案>
//加到最後一次commit
git commit --amend --no-edit
\end{lstlisting}

\subsection{拆掉\code{commit}}
符號\code{\^{}}代表前一次,\code{\^{}\^{}}代表前兩次,以此類推。注意到,\code{reset}指令只是\rtext{回到某版本的狀態},其餘版本都還留存,並沒有刪除。
\begin{lstlisting}[language=git]
//拆掉最後一次commit,以前一次的檔案覆蓋現有檔案
git reset <commit前6碼>^
git reset master^
git reset HEAD^
//指定要拆到第幾個commit
git reset <commit前6碼>
//進入互動模式編輯commit
git rebase -i <SHA-1前七碼>
\end{lstlisting}

\subsection{\code{reset}的三種模式}
\begin{table}[h]
	\centering
	\begin{tabular}{|l|c|c|c|}
		\hline
		模式 & mixed (預設) & soft & hard \\
		\hline
		工作目錄 & 不變 & 不變 & 丟掉 \\
		\hline
		暫存區 & 丟掉 & 不變 & 丟掉 \\
		\hline
		\code{commit}拆出來的檔案 & 丟回工作目錄 & 丟回暫存區 & 完全丟掉 \\
		\hline
	\end{tabular}
	\caption{reset模式}
\end{table}

\subsection{將\code{reset}還原}
已知\code{reset}不刪除檔案,因此只要記得SHA-1的前7碼,完全可以直接\code{reset}回去。若忘記可以使用\code{reflog}指令察看
\begin{lstlisting}[language=git]
git reflog
//這個也可以做到
git log -g
\end{lstlisting}

\section{分支\code{branch}}
第一個\code{commiit}即建立在預設的分支\code{master}上
\subsection{基本操作}
\begin{lstlisting}[language=git]
//新增分支
git branch <分支>
//刪除分支 
git branch -d <branch name> //刪除分支,但合併的不可刪除
git branch -D <branch name> //刪除分支,但強迫刪除未合併分支
//改名
git branch -m <old name> <new name> //幫分支改名
//切換分支
git checkout <branch name>
//合併分支
git merge <要合併的分支> 
\end{lstlisting}

\section{疑難雜症}

\subsection{Git不保存空目錄}
在該目錄中新增\code{.gitkeep}檔案即可

\subsection{讓Git忽略特定檔案}
在與\code{.git}同級的資料夾新增\code{.gitignore}檔案,加入要忽略的檔案即可。
\begin{lstlisting}[language=git]
#忽略特定檔案
<檔名>

#忽略特定副檔名的檔案
*.<副檔名>

#忽略特定資料夾的檔案
<資料夾名稱>
\end{lstlisting}
注意到,在建立.gitignore檔案前就存在的,但在忽略清單中的檔案不會被忽略!可以在建立\code{.gitignore}後,用\code{clean}強制清除:
\begin{lstlisting}[language=git]
git clean -fX
\end{lstlisting}

\section{與GitHub專案同步}
\subsection{push和pull}
首先至GitHub網站創建專案 (repository),然後在本地要同步的,已經受git版控的資料夾加上遠端節點:
\begin{lstlisting}[language=git]
git remote add origin <*.git>
\end{lstlisting}
注意到,遠端節點不一定要命名成origin,這只是一種慣例。
\\\\
然後就可以把本地端的專案推送到Github了。使用\texttt{push}指令推送特定分支到origin。注意到,若origin存在同名分支,即更新進度,不存在則創建分支
\begin{lstlisting}[language=git]
git push -u origin <your branch>
\\
git push origin <本地端分支名> :<遠端分支名>
\end{lstlisting}
\subsection{大型專案常用指令}
沒有存取權時,先去\code{GitHub}\code{fork}一份到自己的\code{repository}
\begin{lstlisting}[language=git]
//複製一份repository到本地
git clone <url>
如何同步fork過來的專案
git remote -v //查看遠端節點有無原作的節點
git remote add <在地節點名> <原作節點url>
git fetch <在地節點名>
git merge <在地節點名>/要合併的節點
//刪除遠端分支
git push origin :<要刪除的分支名>
\end{lstlisting}
\section{圖表示例}

\subsection{如何正確命名分支}
\begin{table}[H]
	\centering
	\begin{tabular}{||l|l||}
		\hline
		分支名稱 & 說明\\
		\hline
		\textbf{Master} & 穩定,可以隨時使用的版本,不會直接commit,版本標籤通常打在此分支上\\
		\hline
		\textbf{Hotfix} & 緊急問題修復開的分支,修好記得merge到Master和Develop。\\
		\hline
		\textbf{Release} & 上線前的測試板,好了一樣要併到Develop,避免日後開發需要\\
		\hline
		\textbf{Feature} & 新增功能時,從Feature拿來的\\
		\hline
	\end{tabular}
	\caption{分支命名範式}
	\label{tab:example}
\end{table}



\subsection{插入圖片}
\begin{figure}[h]
	\centering
	\includegraphics[width=0.5\textwidth]{example-image} % 替換為你的圖片文件
	\caption{這是一個示例圖片}
	\label{fig:example}
\end{figure}
	
\subsection{插入表格}
\begin{table}[h]
	\centering
	\begin{tabular}{|c|c|c|}
		\hline
		A & B & C \\
		\hline
		1 & 2 & 3 \\
		4 & 5 & 6 \\
		7 & 8 & 9 \\
		\hline
	\end{tabular}
	\caption{這是一個示例表格}
	\label{tab:example}
\end{table}
	
\subsection{並排圖表}
\begin{figure}[h]
	\centering
	\begin{subfigure}{0.45\textwidth}
		\centering
		\includegraphics[width=\textwidth]{example-image-a} % 替換為你的圖片文件
		\caption{子圖 A}
	\end{subfigure}
	\hfill
	\begin{subfigure}{0.45\textwidth}
		\centering
		\includegraphics[width=\textwidth]{example-image-b} % 替換為你的圖片文件
		\caption{子圖 B}
	\end{subfigure}
	\caption{並排圖表示例}
	\label{fig:sidebyside}
\end{figure}
	
\section{代碼示例}
\begin{lstlisting}[language=Python, caption=Python 代碼示例]
	def hello_world():
	print("Hello, world!")
	
	hello_world()
\end{lstlisting}
	
\section{結論}
結論打在這裏
	
\end{document}
